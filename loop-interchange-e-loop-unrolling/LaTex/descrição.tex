\section{Descrição da atividade}

A implementação da multiplicação de matrizes quadradas de dimensão 500 foi implementada de 6 maneiras distintas, utilzando alocação estática ou dinâmica e otimizações de \textit{loop interchange}, \textit{loop unrolling} ou sem otimizações.



\begin{figure}[!h]
 \begin{minipage}{0.5\textwidth}
  \centering
  \begin{minted}[fontsize=\footnotesize]{c}
int main()
{
  int a[n * n];
  int b[n * n];
  int c[n * n];
  multiplyNo(c, a, b);
}



  \end{minted}
  \captionof*{listing}{\small Implementação estática}
 \end{minipage}
 \begin{minipage}{0.5\textwidth}
  \centering
  \begin{minted}[fontsize=\footnotesize]{c}
int main()
{
    int *a = malloc(sizeof(int) * n * n);
    int *b = malloc(sizeof(int) * n * n);
    int *c = malloc(sizeof(int) * n * n);
    multiplyNo(c, a, b);
    free(a);
    free(b);
    free(c);
}
  \end{minted}
  \captionof{listing}*{\small Implementação dinâmica}
 \end{minipage}
 % \captionof{listing}{SomeCaption}
  \label{lst:representation_examples}
\end{figure}


\begin{figure}[H]
 \begin{minipage}{0.5\textwidth}
  \centering
  \begin{minted}[fontsize=\footnotesize]{c}
void multiplyNo(int *c, const int *a, 
                const int *b)
{
  for (int i = 0; i < n; i++){
    for (int j = 0; j < n; j++){
      for (int k = 0; k < n; k++){
        c[i*n+j]=a[i*n+k]*b[k*n+j];
      }
    }
  }
}

  \end{minted}
  \captionof*{listing}{\small Multiplicação não otimizada}
 \end{minipage}
 \begin{minipage}{0.5\textwidth}
  \centering
  \begin{minted}[fontsize=\footnotesize]{c}
void multiplyLi(int *c, const int *a, 
                const int *b)
{
  for (int i = 0; i < n; i++){
    for (int k = 0; k < n; k++){
      for (int j = 0; j < n; j++){
        c[i*n+j]=a[i*n+k]*b[k*n+j];
      }
    }
  }
}

  \end{minted}
  \captionof{listing}*{\small \textit{loop interchange}}
 \end{minipage}
 % \captionof{listing}{SomeCaption}
  \label{lst:representation_examples}
\end{figure}

\begin{figure}[H]
\begin{minted}[fontsize=\footnotesize]{c}
void multiplyLu(int *c, const int *a, const int *b)
{
    for (int i = 0; i < n; i++){
        for (int j = 0; j < n; j++){
            int k;
            for (k = 0; k < n / 10;){
                c[i * n + j] = a[i * n + k] * b[k * n + j];
                k++;
                c[i * n + j] = a[i * n + k] * b[k * n + j];
                k++;
                c[i * n + j] = a[i * n + k] * b[k * n + j];
                k++;
                c[i * n + j] = a[i * n + k] * b[k * n + j];
                k++;
                c[i * n + j] = a[i * n + k] * b[k * n + j];
                k++;
                c[i * n + j] = a[i * n + k] * b[k * n + j];
                k++;
                c[i * n + j] = a[i * n + k] * b[k * n + j];
                k++;
                c[i * n + j] = a[i * n + k] * b[k * n + j];
                k++;
                c[i * n + j] = a[i * n + k] * b[k * n + j];
                k++;
                c[i * n + j] = a[i * n + k] * b[k * n + j];
                k++;
            }
            for (; k < n; k++){
                c[i * n + j] = a[i * n + k] * b[k * n + j];
            }
        }
    }
}
\end{minted}
\captionof{listing}*{\small \textit{loop unrolling}}
\end{figure}
Cada experimento realizado é detalhado na tabela abaixo.

\begin{table}[H]
\centering
\begin{tabular}{|c|c|c|c|}
    \hline \textbf{Experimento} & \textbf{Alocação} & \textbf{Otimização} & \textbf{Arquivo implementado} \\
    \hline 1 & Estática & None & \texttt{static-no.c} \\
    \hline 2 & Dinâmica & None & \texttt{dynamic-no.c} \\
    \hline 3 & Estática & \textit{loop interchange} & \texttt{static-li.c} \\
    \hline 4 & Dinâmica & \textit{loop interchange} & \texttt{dynamic-li.c} \\
    \hline 5 & Estática & \textit{loop unrolling} & \texttt{static-lu.c} \\
    \hline 6 & Dinâmica & \textit{loop unrolling} & \texttt{dynamic-lu.c} \\
    \hline
\end{tabular}
\end{table}
    

Os testes foram realizados com o comando \texttt{perf stat} utilizando a flag \texttt{-r} para que cada teste fosse executado diversas vezes, \texttt{-e} para coletar informações sobre os eventos desejados e \texttt{-o} para armazenar os resultados no arquivo informado. Dessa forma, o comando abaixo foi usada para executar o teste com alocação estática e \textit{loop interchange}, por exemplo.

\begin{minted}{shell}
perf stat -r 10  -o ../perf-out/static-li.txt -e L1-dcache-loads,
L1-dcache-load-misses,branch-instructions,branch-misses
../bin/static-li
\end{minted}

O script \texttt{make\_csv.py} foi usado para processar as saídas geradas e compilar as informações dos testes no arquivo \texttt{experiment\_results.csv}.

Os intervalos de confiança do tempo foram calculados com o script \texttt{calculate} \texttt{\_confidence\_intervals.py}, utilizando a bilbioteca \texttt{scipy.stats}. Assumiu-se que o tempo segue a distribuição \textit{t-sudent} e utilizou-se um p-valor de 5\%. O erro padrão da média (SEM) de cada experimento foi calculado com base no desvio padrão ($\sigma$) informado pelo comando \texttt{perf stat} e pela quantidade de medidas tomadas (n), nesse caso 10, por meio da fórmula $SEM = \sigma / \sqrt{n}$. Os resultados foram registrados no arquivo \texttt{experiment\_results.csv}.

\begin{minted}{python}
static_no_sem = static_no_stddev / np.sqrt(len)
st.t.interval(confidence=0.95, df=len-1, loc=static_no_mean, 
              scale=static_no_sem)
\end{minted}

A influência dos fatores foram calculadas com o script \texttt{calculate\_factors\_} \texttt{influence.r}, desenvolvido com base no script disponíbilizado, e o resultado foi salvo no arquivo \texttt{factors\_influence.csv}. Foram realizadas 4 análises, todas do tipo 2 fatorial completo, descritas na tabela abaixo.  

\begin{table}[H]
\centering
\begin{tabular}{|c|c|c|c|c|c|}
    \hline Análise & \multicolumn{2}{c|}{\textbf{Nível alocação}} & \multicolumn{2}{c|}{\textbf{Nível otimização}} & \textbf{Variável resposta} \\
    \hline \textbf{1} & Estática & Dinâmica & None & \textit{interchange} & L1-dcache-loads \\
    \hline \textbf{2} & Estática & Dinâmica & None & \textit{interchange} & L1-dcache-loads-misses \\
    \hline \textbf{3} & Estática & Dinâmica & None & \textit{unrolling} & branch-instructions \\
    \hline \textbf{4} & Estática & Dinâmica & None & \textit{unrolling} & branch-misses \\
    \hline
\end{tabular}
\end{table}
    
% A multiplicação de matrizes foi implementada sem otimizações, utilizando \textit{loop unrolling} e \textit{loop interchange}. O tamanho das matrizes utilizadas durantes os testes foi de 500 x 500.

% \begin{minted}{c}
% void multiplyNo(int *c, const int *a, const int *b) {
%   for (int i = 0; i < n; i++) {
%     for (int j = 0; j < n; j++) {
%       for (int k = 0; k < n; k++) {
%         c[i * n + j] = a[i * n + k] * b[k * n + j];
%       }
%     }
%   }
% }
% \end{minted}

% \begin{minted}{c}
% void multiplyLu(int *c, const int *a, const int *b) {
%   for (int i = 0; i < n; i++) {
%     for (int j = 0; j < n; j++) {
%       int k;
%       for (k = 0; k < n / 10;) {
%         c[i * n + j] = a[i * n + k] * b[k * n + j];
%         k++;
%         c[i * n + j] = a[i * n + k] * b[k * n + j];
%         k++;
%         c[i * n + j] = a[i * n + k] * b[k * n + j];
%         k++;
%         c[i * n + j] = a[i * n + k] * b[k * n + j];
%         k++;
%         c[i * n + j] = a[i * n + k] * b[k * n + j];
%         k++;
%         c[i * n + j] = a[i * n + k] * b[k * n + j];
%         k++;
%         c[i * n + j] = a[i * n + k] * b[k * n + j];
%         k++;
%         c[i * n + j] = a[i * n + k] * b[k * n + j];
%         k++;
%         c[i * n + j] = a[i * n + k] * b[k * n + j];
%         k++;
%         c[i * n + j] = a[i * n + k] * b[k * n + j];
%         k++;
%       }
%       for (; k < n; k++) {
%         c[i * n + j] = a[i * n + k] * b[k * n + j];
%       }
%     }
%   }
% }
% \end{minted}

% \begin{minted}{c}
% void multiplyLi(int *c, const int *a, const int *b) {
%   for (int i = 0; i < n; i++) {
%     for (int k = 0; k < n; k++) {
%       for (int j = 0; j < n; j++) {
%         c[i * n + j] = a[i * n + k] * b[k * n + j];
%       }
%     }
%   }
% }
% \end{minted}

% Também, a alocação da memória para as matrizes foi implementada estaticamente e dinamicamente, nos arquivos \texttt{static.c} e \texttt{dynamic.c}.

% \begin{minted}{c}
% int main()
% {
%   int a[n * n];
%   int b[n * n];
%   int c[n * n];
%   multiplyNo(c, a, b);
% }
% \end{minted}

% \begin{minted}{c}
% int main()
% {
%     int *a = malloc(sizeof(int) * n * n);
%     int *b = malloc(sizeof(int) * n * n);
%     int *c = malloc(sizeof(int) * n * n);
%     multiplyNo(c, a, b);
%     free(a);
%     free(b);
%     free(c);
% }
% \end{minted}

% O arquivo \texttt{static.c} foi compilado para \texttt{static-no}, \texttt{static-lu} e \texttt{static-li}, utilizando as diferentes estratégias de multiplicação de matrizes. Analogamente para o arquivo \texttt{dynamic.c}. Os experimentos foram realizados utilizando o comando abaixo para multiplicação de uma matriz estática utilizando a otimização \textit{loop interchange}, por exemplo.

% \begin{minted}{shell}
% perf stat -r 20 -e L1-dcache-loads,L1-dcache-load-misses,
% branch-instructions,branch-misses ./static-li
% \end{minted}

% Desse modo, temos 6 experimentos, onde cada um deve ser repetido 10 vezes, a saber: 

% \begin{itemize}
% \item Sem otimização com alocação estática
% \item Sem otimização com alocação dinâmica
% \item Loop Interchange com alocação estática
% \item Loop Interchange com alocação dinâmica
% \item Loop Unrolling com alocação estática
% \item Loop Unrolling com alocação dinâmica
% \end{itemize}
 